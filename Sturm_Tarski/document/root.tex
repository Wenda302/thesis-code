\documentclass[11pt,a4paper]{article}
\usepackage{isabelle,isabellesym}
\usepackage{amsmath}
\usepackage{amssymb} 

% this should be the last package used
\usepackage{pdfsetup}

% urls in roman style, theory text in math-similar italics
\urlstyle{rm}
\isabellestyle{it}


\begin{document}

\title{Sturm-Tarski Theorem}
\author{Wenda Li}
\maketitle

\begin{abstract}
  We have formalized the Sturm-Tarski theorem (also referred as the Tarski theorem): \[\sum_{x \in (a,b),P(x)=0} {\rm sign}(Q(x)) = {\rm Var}({\rm SRemS}(P,P'Q;a,b))\] 
  where $a<b$ are elements of $\mathbb{R} \cup \{-\infty,\infty\}$ that are not roots of $P$, with $P,Q \in \mathbb{R}[x]$. Note, 
  the usual Sturm theorem is an instance of the Sturm-Tarski theorem with $Q=1$. The proof is based on \cite{Basu:2006:ARA:1197095} and Cyril Cohen's work in Coq \cite{cohen_phd}. With the Sturm-Tarki theorem proved, it is possible to further build a quantifier elimination procedure for real numbers as Cyril Cohen does in Coq.
  Another application of the Sturm-Tarski theorem is to build sign determination procedures for polynomials at real algebraic points, as described in our formalization of real algebraic numbers \cite{Li_CPP_16}.
\end{abstract}

%\tableofcontents

% include generated text of all theories
\input{session}

\bibliographystyle{abbrv}
\bibliography{root}

\end{document}
